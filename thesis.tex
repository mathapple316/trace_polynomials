\documentclass[12pt,reqno,letter,oneside,openright]{book}

%%%%%%%%%%%%
%\usepackage[euc]{kotex}
\usepackage{kotex}               % 한글 설정
%%%%%%%%%%%
\usepackage{upgreek}
%\usepackage[notcite, notref]{showkeys}
%\usepackage[active]{srcltx}
\usepackage{amsthm,amsxtra}
\usepackage{amssymb,amsmath,esint}
\usepackage{amsfonts}
\usepackage[poly,all]{xy}
\usepackage[ruled,vlined]{algorithm2e}
\usepackage{fancyhdr}
\usepackage{snumaththesis1}
\usepackage[CJKbookmarks,bookmarks=true,colorlinks=true]{hyperref}
\usepackage{amssymb, latexsym}  
\usepackage{mathrsfs}
\usepackage{tikz}
\usepackage{graphicx}
\usepackage{algpseudocode}
\graphicspath{ {images/} }
\usepackage{enumerate}
\usepackage{import}
\usepackage{verbatim}
\usepackage[doublespacing]{setspace}
\usepackage{colortbl}
\usepackage{longtable}
\usepackage{cite}
\usepackage{listings}
\usepackage{xcolor}
\include{pythonlisting}
%\usepackage{subfig}
%\usepackage{longtable}
%\usepackage[inline]{enumitem}
\usepackage{tikz-cd}
%\usepackage{imakeidx}
%\makeindex[intoc]

\allowdisplaybreaks

%%%%%%%%%%%%%%%%%%%%%%%%%%%%%%%%%%%%%%%%%%%%%%%%%%%%%%%%%%%%%%%%%%%
\usepackage{hyperref}
\colorlet{linkequation}{blue}
%\usepackage[colorlinks,plainpages=true,pdfpagelabels,hypertexnames=true,colorlinks=true,pdfstartview=FitV,linkcolor=blue,citecolor=red,urlcolor=black]{hyperref}
\PassOptionsToPackage{unicode}{hyperref}
\PassOptionsToPackage{naturalnames}{hyperref}
%%%%%%%%%%%%%%%%%%%%%%%%%%%%%%%%%%%%%%%%%%%%%%%%%%%%%%%%%%%%%%%%%%%

%%%%%%custom command%%%%%%%%%%
\newcommand{\bigslant}[2]{{\raisebox{.2em}{$#1$}\left/\raisebox{-.2em}{$#2$}\right.}}
\newcommand{\cw}{\mathcal{C}}
\newcommand{\cwn}{\mathcal{C}_n}
\newcommand{\tr}{\operatorname{tr}}
\newcommand{\Mir}{\operatorname{Mir}}
\renewcommand{\Lsh}{\operatorname{Lsh}}
\newcommand{\Inv}{\operatorname{Inv}}
\newcommand{\TR}{\operatorname{\alpha}}
\newcommand{\TRW}[1]{\operatorname{\alpha}(#1)}
\newcommand{\norm}[1]{\left\lVert#1\right\rVert}
\newcommand{\oo}{\newline\noindent}
\newcommand{\iiii}{\indent}
\newcommand{\len}[1]{\mid #1 \mid}
\newcommand{\slen}[1]{{\mid #1 \mid}_{syl}}
\newcommand{\clen}[1]{{\mid #1 \mid}_{csyl}}
\newcommand{\alp}[2]{\gamma_{#1}(#2)}
\newcommand{\ssl}{\operatorname{SL}(2,\mathbb{C})}
\everymath{\displaystyle}
%\usepackage[inline]{showlabels}
%\usepackage[notcite,notref]{showkeys}
%\usepackage[left]{lineno}
%\linenumbers
\renewcommand{\baselinestretch}{1.1}

\hypersetup{%
pdftitle={Master's thesis},%
pdfauthor={Author, Department of Mathematical Sciences, Seoul National University, Republic of Korea},%
pdfkeywords={pdfkeywords,pdfkeywords,pdfkeywords},%
citecolor=black,%
filecolor=black,%
linkcolor=black,%
urlcolor=black,%
}


\numberwithin{equation}{chapter}
%\everymath{\displaystyle}


%%%%% THEOREM STYLES %%%%%

\newtheorem{theorem}{Theorem}[section]
\newtheorem{theorem*}{Theorem}
\newtheorem{prop}[theorem]{Proposition}
\newtheorem{proposition}[theorem]{Proposition}
\newtheorem{lemma}[theorem]{Lemma}
\newtheorem{cor}[theorem]{Corollary}
\newtheorem{corollary}[theorem]{Corollary}
\newtheorem*{conjecture*}{Conjecture of Wang}
\theoremstyle{definition}
\newtheorem{definition}[theorem]{Definition}
\newtheorem{remark}[theorem]{Remark}%[section]
\newtheorem{example}{Example}

%%%%%%%%%%%%%%%%%%%%%%%%%%%%%%%%%%%%%%%%%%%%%%%%%%%%%%%%%%%%%%%%%%%%%%%%%%%%%%%%%%%%%%%%%%%


%%%%% EQUATIONS %%%%%

%\def\eqn#1$$#2$${\begin{equation}\label#1#2\end{equation}}
%\def\charfn_#1{{\raise1.2pt\hbox{$\chi_{\kern-1pt\lower3pt\hbox{{$\scriptstyle#1$}}}$}}}
%\newcommand{\rif}[1]{(\ref{#1})}
%\newcommand{\trif}[1] {\textnormal{\rif{#1}}}
%\newcommand{\tref}[1] {\textnormal{\ref{#1}}}
%\makeatletter
%\newcommand{\pushright}[1]{\ifmeasuring@#1\else\omit\hfill$\displaystyle#1$\fi\ignorespaces}
%\newcommand{\pushleft}[1]{\ifmeasuring@#1\else\omit$\displaystyle#1$\hfill\fi\ignorespaces}
%\makeatother



%%%%% GREEK LETTERS %%%%%

%%%%% OPERATORS %%%%%

%%%%% SHORTCUTS %%%%%

%%%%% INTEGRALS %%%%%


\setlength{\headheight}{15pt}





%%%%%%%%%%%%%%%%%%%%%%%%%%%%%%%%%%%%%%%%%%%%%%%%%%%%%%%%%%%%%%%%%%%%%%%%%%%%%%%%%%%%%%%%%%%%%%%%%%%%%%%%%%%%%%%%%%%%%%%%%%%%
\bibliographystyle{amsplain}
\AtBeginDocument{%
   \def\MR#1{}
}

%%%%%%%%%%%%%%%%%%%%%%%%%%%%%%%%%%%%%%%%%%%%%%%%%%%%%%%%%%%%%%%%%%%%%%%%%%%%%%%%%%%%%%%


\begin{document}

%%%%%%%%%%%%%%%%%%%%%%%%%%%%%%%%%%%%%%%%%%%%%%%%%%%%%%%%%%%%%%%%%%%%%%%%%%%%%%%%%%%%%%%%%%%%%%%%%%%%%%%%%%%%%%%%%%%%%%%%%%%%%%%%%%%%%%%%%%%%%%%%
%%%%%%%%%%%%%%%%%%%%%%%%%%%%%%%%%%%%%%%%%%%%%%%%%%%%%%%%%%%%%%%%%%%%%%%%%%%%%%%%%%%%%%%%%%%%%%%%%%%%%%%%%%%%%%%%%%%%%%%%%%%%%%%%%%%%%%%%%%%%%%%%


%% 본문 이전의 쪽들

\doublespacing
\frontmatter


%%%%%%%%%%%%%%%%%%%%%%%%%%%%%%%%%%%%%%%%%%%%%%%%%%%%%%%%%%%%%%%%%%%%%%%%%%%%%%%%%%%%%%%%%%%%%%%%%%%%%%%%%%%%%%%%%%%%%%%%%%%%%%%%%%%%%%%%%%%%%%%%

%%% 기초 정보 입력

%% 영어 제목. 어쩔 수 없는 경우가 아니라면 수식 등을 사용하지 않는다.
\EnglishTitle{Trace polynomials of words in the free group of rank two}

%% 국어 제목. 어쩔 수 없는 경우가 아니라면 수식 등을 사용하지 않는다.
\KoreanTitle{계수 2 자유군에서의 대각합 다항식}

%% 지은이
\AuthorKoreanName{박현수}

%% 지은이 영어 이름
\AuthorEnglishName{Hyeonsu Park}

%% 박사과정 학번
\StudentNumber{201520262}

%% 국문학위명. 이학 박사 / 이학 석사
\KoreanNameofDegree{이학석사}

%% 영문학위명. Doctor of Philosophy / Master of Science
\EnglishNameofDegree{Master of Science}

%% 날짜 1학기 / 2학기
%% 학위 수여일. 2월 / 8월
\GraduateDate{2020년 8월} %
\GraduateDateEnglish{August 2020}

%% 심사용 논문 제출 기한. 4월 / 10월
\SummittedDate{2020년 4월}

%% 종심 끝난 날짜. 6월 / 12월
\RefereeDate{2020년 7월}

%% 논문 심사위원
\RefereeChief{} \RefereeSecond{} \RefereeThird{}

%% 지도교수 성명
\AdvisorKoreanName{임선희} \AdvisorEnglishName{Seonhee Lim}

%% Keywords. keyword는 논문을 검색할 때 색인어로 가능한 단어를 6개 이내로 선정
\KeyWordsEnglish{Trace polynomial, Free group of rank two, Special linear group, Cyclically reduced words}
\KeyWordsKorean{대각합 다항식, 계수2 자유군, 특수 선형군, 순환 기약 워드}%

%%%%%%%%%%%%%%%%%%%%%%%%%%%%%%%%%%%%%%%%%%%%%%%%%%%%%%%%%%%%%%%%%%%%%%%%%%%%%%%%%%%%%%%%%%%%%%%%%%%%%%%%%%%%%%%%%%%%%%%%%%%%%%%%%%%%%%%%%%%%%%%%

%%% 각종 양식 만들기
%%%
%%% 최종 제출 버젼이 아니라면 각종 양식은 필요없다.
%%% 이 경우에는 `각종 양식 만들기' 부분을 전부 주석처리하면 된다.

%% 앞표지 만들기



\makefrontcover

%% 국문 제출 승인서/인준지 만들기
\makeapproval

%% 영문 제출 승인서
\makeenglishapproval

\singlespacing

%% 학위논문 원문제공 서비스에 대한 동의서 만들기 (이전 양식인듯)
%% 논문
%\makewrittenconsent

%% ``the Copyright"쪽 만들기
\makecopyright{2020}

%%%%%%%%%%%%%%%%%%%%%%%%%%%%%%%%%%%%%%%%%%%%%%%%%%%%%%%%%%%%%%%%%%%%%%%%%%%%%%%%%%%%%%%%%%%%%%%%%%%%%%%%%%%%%%%%%%%%%%%%%%%%%%%%%%%%%%%%%%%%%%%%

%% 영문초록.
%%
%% 1. 논문의 내용과 결론에 관하여 간략하고 구체적으로 기재
%%
%% 2. 스스로 정의한 수식, 명령어 등의 매크로를 사용하지 않는다.
%%    영문초록은 다음과 같은 최소한의 설정 아래에서도
%%    아무런 문제없이 컴파일이 되어야 한다.
%%
%%    \documentclass{article}
%%    \usepackage{최소한의 패키지 사용}
%%    \begin{document}
%%    영문초록 내용
%%    \end{document}
%%
%%    그래야만 인터넷 등의 자동화 된 시스템에서 다른 이들도
%%    이 논문의 초록을 제대로 읽을 수 있기 때문이다.
%%
%% 3. 영문초록에서 논문의 각 chapter, section의 내용을 요약할 필요는 없다.
%%    왜냐하면 chapter, section의 내용 요약은 Introduction chapter에서 해야 하는 것이기 때문이다.

\pagenumbering{roman} \setcounter{page}{1}

\begin{EnglishAbstract}
\hspace{\parindent}
Procesi's theorem guarantees that traces in a two generator subgroup of $\ssl$ are polynomials in traces of the generators. These polynomials are called trace polynomials and defined for words in the free group of rank two. Let $\cw$ denote the set of cyclically reduced words in $F_2$. 
Improving Jorgensen's algorithm, we classify all words in $\cw$ with the word lengths less than nine via their trace polynomials. Then we check whether they are in $\sim$-equivalence defined from the operation Mirror, Left shift, and Inverse on $\cw$.
We prove that two words of the same trace polynomials are $\sim$-equivalent when the word lengths are less than nine.
We also show, by counterexamples, this result does not hold for the word lengths greater than eight. As a corollary, we verify Wang's conjecture for the word lengths less than nine.
\end{EnglishAbstract}

%\newpage\null\thispagestyle{empty}

%%%%%%%%%%%%%%%%%%%%%%%%%

\pagestyle{fancy}%
\lhead{\leftmark}%
\rhead{}%
\chead{}%
\cfoot{\thepage}%
\renewcommand{\headrulewidth}{0pt}


\tableofcontents

%%%%%%%%%%%%%%%%%%%%%%%%%%%%%%%%%%%%%%%%%

\mainmatter

\pagenumbering{arabic} \setcounter{page}{1}
%%%%%%%%%%%%%%%%%%%%%%%%%%%%%%%%%%%%%%%%%%
%%%%%%%%%%

    % 표목차 (List of Tables) 생성
%    \listoftables

    % 그림목차 (List of Figures) 생성
%    \listoffigures

    % 위의 세 종류의 목차는 한꺼번에 다음 명령으로 생성할 수도 있습니다.
    %\makecontents

%% 이하의 본문은 LaTeX 표준 클래스 report 양식에 준하여 작성하시면 됩니다.
%% 하지만 part는 사용하지 못하도록 제거하였으므로, chapter가 문서 내의
%% 최상위 분류 단위가 됩니다.
%% You cannot use 'part'

\import{modules/}{Introduction}

\import{modules/}{Preliminaries}

\import{modules/}{Computation}

\import{modules/}{Mainresults}

% 2~8에 대한 테이블
\import{modules/}{Conclusion}

\import{modules/}{tables}


\newpage
\thispagestyle{empty}%
    \phantomsection
\backmatter

\addcontentsline{toc}{chapter}{Bibliography}
 %%%%퍼옴
\begin{thebibliography}{3}
\bibitem{jorgensen} 
T.Jorgensen:
\emph{Traces in 2-Generator Subgroups of $\mathrm{SL}(2,\mathbf{C})$} 
Proceedings of the American Mathematical Society, Vol. 84, No. 3 (Mar., 1982), pp.339-343

\bibitem{wang} 
Wang Guizhen:
\emph{A Note on Trace Polynomial}
Tsinghua science and technology (2007)
 
\bibitem{triana} 
Charles R. Traina:
\emph{Trace polynomial for two generator subgroups of SL(2, Q)}
AMS Volume 79, Number 3 (1980)
 
\bibitem{Goldman} 
William M. Goldman:
\emph{Trace coordinates on Fricke spaces of some
simple hyperbolic surfaces}
EMS Volume2 Publishing 2009, 611-864
 
\bibitem{Southcott} 
J.B.Southcott
\emph{Trace Polynomials of words in special linear groups}
Austral. Math. Soc. (Series A) 28 (1979), 401-412

\bibitem{Procesi} 
Procesi, C.,
\emph{The invariant theory of n × n matrices}
Adv. Math. 19 (1976), 306–381.
\end{thebibliography}
\newpage
%%%%퍼옴끝

\begin{KoreanAbstract}
\par 
\setstretch{1.3}
$\ssl$의 생성자가 $2$개인 부분군에서는, 행렬대각합을 생성자들의 대각합에 대한 다항식으로 나타낼 수 있고, 그 다항식은 생성자들의 선택에 의존하지 않는다. 이 다항식을 대각합 다항식이라 한다. 본 학위논문에서는 Jorgensen의 정리로 대각합 다항식을 계산하는 알고리듬을 구현하였고, 이 결과로 워드길이 $8$ 이하의 모든 순환기약워드들을 분류하였다. 다음으로, 분류한 순환기약워드들이 Mir, LeftShift, Inverse로 정의되는 $\sim$ 동치관계에 있는지를 조사하였다. 이것으로 다음 두 가지를 증명하였다. 먼저 워드길이 $8$ 이하에 대해, 두 순환기약워드가 $\sim$-동치관계인것은 대각합다항식이 같은것과 동치임을 보였다. 다음으로 워드길이 $9$ 이상부터는, 대각합 다항식은 같지만 $\sim$-동치관계는 아닌 두 순환기약워드가 존재함을 보였다. 이것은 워드길이 $8$이하에 대해, Wang의 
추측이 성립한다는 것을 보여준다.
\end{KoreanAbstract}

\end{document}
