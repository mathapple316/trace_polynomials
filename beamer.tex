\documentclass[aspectratio={169}]{beamer}
\usefonttheme{professionalfonts}
\usefonttheme[onlymath]{serif}
\usecolortheme{orchid}
\useinnertheme[shadow=true]{rounded}
\useinnertheme[shadow=true]{rounded}
\useinnertheme{rectangles}
\usepackage{amsthm,amsxtra}
\usepackage{amssymb,amsmath,esint}
\usepackage{amsfonts}
\usepackage{bbding}
\usepackage{amssymb, latexsym}  
\usepackage{mathrsfs}
\usepackage{kotex}
\usepackage{tikz,tikz-cd}
\usetikzlibrary{arrows}
\allowdisplaybreaks
\usepackage{tikz}
\usetikzlibrary{calc}
\usepackage{mathtools}
\usepackage{hf-tikz}
\usepackage{listings}
\usepackage{graphicx}
\graphicspath{ {./images/} }


\setbeamercolor*{block title example}{fg=red!50!black,bg= green!80!black!50}
\setbeamercolor*{block body example}{fg=green!20!black, bg= green!15}
\setbeamercolor*{block body alerted}{fg= orange!50!black, bg= orange!15}
\setbeamercolor*{block title alerted}{fg=yellow!50, bg= orange!50!red}
\usepackage[skins]{tcolorbox}

\mode<presentation>
{
  \usetheme{Frankfurt}
  \usecolortheme{default} % or try albatross, beaver, crane, ...
  \usefonttheme{default}  % or try serif, structurebold, ...
  \setbeamertemplate{navigation symbols}{}
  \setbeamertemplate{caption}[numbered]
  \setbeamertemplate{frametitle continuation}{}
}   

\defbeamertemplate{itemize item}{boldarrow}{\raisebox{0.1ex}{\resizebox{1.5ex}{1.5ex}{\ArrowBoldRightShort}}}

\usepackage[english]{babel}
\usepackage[utf8]{inputenc}
\usepackage[T1]{fontenc}

\title{Trace polynomials of words in $F_2$}
\subtitle
{Master's thesis presentation}
\author{Hyeonsu Park}
%\author{2015-20262 박현수}
\institute{Department of Mathematical Sciences, SNU}
\date{May 22, 2020}

\begin{document}

\begin{frame}
  \titlepage
\end{frame}

% Uncomment these lines for an automatically generated outline.
%\begin{frame}{Outline}
%  \tableofcontents
%\end{frame}

\newcommand{\bigslant}[2]{{\raisebox{.2em}{$#1$}\left/\raisebox{-.2em}{$#2$}\right.}}
\newcommand{\cw}{\mathcal{C}}
\newcommand{\cwn}{\mathcal{C}_n}
\newcommand{\tr}{\operatorname{tr}}
\newcommand{\Mir}{\operatorname{Mir}}
\newcommand{\zxyz}{\mathbb{Z}[x,y,z]}
\renewcommand{\Lsh}{\operatorname{Lsh}}
\newcommand{\Inv}{\operatorname{Inv}}
\newcommand{\TR}{\operatorname{\alpha}}
\newcommand{\TRW}[1]{\operatorname{\alpha}(#1)}
\newcommand{\norm}[1]{\left\lVert#1\right\rVert}
\newcommand{\oo}{\newline\noindent}
\newcommand{\iiii}{\indent}
\newcommand{\len}[1]{\mid #1 \mid}
\newcommand{\slen}[1]{{\mid #1 \mid}_{syl}}
\newcommand{\clen}[1]{{\mid #1 \mid}_{csyl}}
\newcommand{\alp}[2]{\gamma_{#1}(#2)}
\newcommand{\ssl}{\operatorname{SL}(2,\mathbb{C})}
\everymath{\displaystyle}

%%%%%%%%%%%%%%%%%%%%%%%%%%%%%%%%%%%%%%%%%%%%%%%%%%%%%%%%%%%%%%%%

\AtBeginSection[]
{
\begin{frame}{Contents}
\tableofcontents[currentsection]
\end{frame}
}

\section{Introduction}
\subsection{Traces in $\ssl$}
\begin{frame}{Traces in $\ssl$}
Let $A \in \ssl$ and let $\tau = \tr(A)$. Then $A+A^{-1} = \tau I$.
\vskip 0.5cm
\begin{itemize}
  \item $A^2 = \tau A - I$
\vskip 0.5cm
  \item $A^3 = (\tau^2 - 1)A - \tau I $
\vskip 0.5cm
  \item $A^4 =  (\tau^3-2\tau)A - (\tau^2 -1)I$\\
  \hspace{2cm} \vdots
  \pause
\setbeamertemplate{itemize item}[boldarrow]
\item
$A^{n} =  \beta_n(\tau)A - \beta_{n-1}(\tau)I$
\vskip 0.2cm
where $\beta_{n+1}(x)=x \beta_n(x) - \beta_{n-1}(x)$ with $\beta_0(x) = 0$ and $\beta_1(x)=1$.
\vskip 0.1cm
(Chebyshev polynomials)
\end{itemize}
\end{frame}

\begin{frame}{Traces in $\ssl$}
\setbeamertemplate{itemize item}[boldarrow]
For $A\in\ssl$, we have
\begin{equation*}
\tr(A^n) = \tau\beta_{n}(\tau) - 2\beta_{n-1}(\tau)
=: \tau_n(\tr A).
\end{equation*}
\begin{itemize}
\item[1] Traces of $A^n$ is are polynomials in $\tr A$.
\vskip 0.2cm
\item[2] $\{\tau_n\}$ are only dependent on $n$, not on the choice of $A$.
\vskip 1cm
\pause
\item What about the case of two-generator subgroups?
\end{itemize}
\end{frame}
%%%%%%%%%%%%%%%%%%%%%%%%%%%%%%%%%%%%%%%%%%
\subsection{Trace polynomial $\TR$}
\begin{frame}{Trace polynomial $\TR$}
Procesi's theorem gives us the answer.
\begin{theorem}[Procesi]
Let $f : \ssl \times \ssl \longrightarrow \mathbb{C}$ be a regular function. Moreover, let $f$ be invariant under the diagonal action of $\ssl$ by conjugation. Then there exists $F \in \mathbb{Z}[x,y,z]$ such that
\begin{align*}
 f(A,B) = F(\tr A,\tr B,\tr AB) 
\text{ for all } A,B \in \ssl.
\end{align*}
\end{theorem}
\end{frame}
%%%%%%%%%%%%%%%%%%%%%%%%%%
\begin{frame}{Trace polynomial $\TR$}
\begin{itemize}
\item \text{For $A,B\in\ssl$, consider a group homomorphism}
\begin{align*}
\rho_{A,B} \text{ : } &F_{2}\rightarrow \ssl\\
&a \mapsto A, \\
&b \mapsto B. 
\end{align*}
\vskip-10ex
\setbeamertemplate{itemize item}[boldarrow]
\item For $w\in F_2$, apply Procesi's theorem to 
\vskip-2ex
\begin{equation*}
f_w(A,B) = \tr\Big(\rho_{A,B}(w) \Big) \text{ for all } A,B \in \ssl.
\end{equation*}
\end{itemize}
\pause
\begin{block}{Definition of Trace polynomial $\TR(w)$}
For $w \in F_2$, there exists the unique polynomial $\TR(w) \in \zxyz$ such that
\begin{equation*}
\tr \Big( \rho_{A,B}(w) \Big) = \TR(w) \Big( \tr A, \tr B, \tr AB \Big) \text{ for all } A,B \in \ssl.
\end{equation*}
\end{block}
\end{frame}
%%%%%%%%%%%%%%%%%%%%%%%%%%%%%%%%%%%%%%%%%%%%%%%%%%%%%
\begin{frame}{Trace polynomial $\TR$}
\begin{itemize}
\item The trace polynomial of $w$ is unique because of the surjectivity of
\begin{align*}
    \ssl \times \ssl &\rightarrow \mathbb{C}^3 
    \text{ given by }\\
    (A,B) &\mapsto (\tr A, \tr B, \tr AB).
\end{align*}
\item For $v,w \in F_2$, we see that \begin{align*}
\TR(wv)+\TR(w v^{-1})=\TR(w)\TR(v) \text{ and } \TR(w)=\TR(w^{-1}).
\end{align*}
\item For $f\in\opertorname{End}(F_2)$, we have
\begin{equation*}
\TR(f(w)) = \TR(w)\Big( \TR\big(f(a) \big), \TR \big( f(b) \big), \TR \big( f(ab) \big) \Big) \text{ for } w\in F_2.
\end{equation*}
\end{itemize}
\end{frame}
%%%%%%%%%%%%%%%%%%%%%%%%%%%%%%%%%%%%%%%%%%%%%%%%%%%%%%
%%%%%%%%%%%%%%%%%%%%%%%%%%%%%%%%%%%%%%%%%%%%%%%%%%%%%%
\begin{frame}{Examples}
\begin{itemize}
    \item  $w = abab^{-1}$
\end{itemize}
\begin{align*}
\TRW{abab^{-1}} &= \TRW{aba} \TRW{b^{-1}} - \TRW{abab} \\ 
&= y\TRW{ba^2} - \TR \big( (ab)^2 \big) \\
&= y(\TRW{ba}\TRW{a} - \TRW{b}) - (\TRW{ab}\TRW{ab} - \TRW{e}) \\
&= y(xz-y) - (z^2 - 2) = 2 - y^2 - z^2 + xyz.
\end{align*}
\end{frame}

\begin{frame}{Examples}
\begin{itemize}
    \item  $w = aba^3b^{-1}$
\end{itemize}
\vskip 0.5cm
Consider $f\in \opertorname{End(F_2)}$ given by $f(a)=a$ and $f(b)=bab^{-1}$. 
\\Then $f(ab^3)$ is equal to $w$.
\begin{align*}
\TRW{f(w)} 
&= 
\TR(w)\Big( \TR\big(f(a) \big), \TR \big( f(b) \big), \TR \big( f(ab) \big) \Big) \\
&= \TRW{ab^{3}} \Big( \TRW{a}, \TRW{bab^{-1}}, \TRW{abab^{-1}} \Big) \\
&= (y^2z - xy - z) \Big(x, x, 2-y^2-z^2+xyz \Big) \\
&= -2 + x^2+y^2+z^2 - xyz - x^2y^2 - x^2z^2 + x^3yz 
\end{align*}
\end{frame}
%%%%%%%%%%%%%%%%%%%%%%%%%%%%%%%%%%%%%%%%%%%%%%%%%%%%%%
\section{Computation of trace polynomials}
\subsection{A naive algorithm}
\begin{frame}{Notation}
Let $m\in \mathbb{Z}^{2r}$ for $r>0$. \\
\begin{itemize}
    \item 
For $g,h$ in a multiplicative group $G$, define 
\begin{equation*}
    m(g,h): = \prod_{i=1}^{r}{g^{m_{2i-1}} h^{m_{2i}} } \in G.
\end{equation*}
$m=(1,3,2,4)$ and $F_2=\langle a,b \rangle$ $\Longrightarrow  m(a,b)=ab^3a^2b^4$.
\vskip 0.5cm
\pause
\item
Define $\mathbf{B}(m) \in \mathbb{Z}[x,y]$ by
\begin{equation*}
 \mathbf{B}(m) := \prod_{i=1}^{r}{\beta_{m_{2i-1}}(x) \beta_{m_{2i}}(y) } 
\end{equation*}
$m=(1,3,2,4) \Longrightarrow \mathbf{B}(m)=\beta_1(x)\beta_3(y)\beta_2(x)\beta_4(y)$.
\end{itemize}
\end{frame}
%%%%%%%%%%%%%%%%%%%%%%%%%%%%%%%%%%%%%%%%%%%%%%%%%%%%%%
\begin{frame}{Notation}
For $m \in \mathbb{Z}^{2r}$, $\TR(m)$ denotes the trace polynomial of $m(a,b)$, that is
\begin{equation*}
    \TR(m) := \TR \Big( m(a,b) \Big).
\end{equation*}
\vskip 0.5cm
Examples \\
\begin{itemize}
    \item  $\TR(1,1,1,1) = \TR(abab) = z^2-2 $
    \item  $\TR(1,1,0,1) = \TR(ab^2) = yz - x $
\end{itemize}
\end{frame}
%%%%%%%%%%%%%%%%%%%%%%%%%%%%%%%%
\begin{frame}{A naive algorithm}
$\TR(m)
= \sum_{{ {\epsilon \in {\{0,1\}}^{2r}} }}(-1)^{\norm{\epsilon}_1}\cdot\textbf{B}(\mathbf{\mathbf{m} + \epsilon - \mathbf{1}})(x,y)\cdot\textcolor{blue}{\TR(\epsilon)}$ \\
\vskip 0.2cm
Suffices to compute $\TR(\epsilon)$ for all $\epsilon \in { \{ {0,1 \} }^{2r}}$.
\begin{itemize}
\item $\epsilon = (1,1,1,1,\dots,1,1)$ \\
\vskip 0.3cm
$\Longrightarrow \TR(\epsilon) = \TR \big( (ab)^r \big) = \tau_r{(z)}$
\vskip 0.5cm
\item $\epsilon$ has at least one zero entry. \\
\vskip 0.3cm
$\Longrightarrow$ Reduce $\epsilon$ into $\epsilon'\in { {\{ 0,1 \}}^{2r} }$ so that $\epsilon(a,b)=\epsilon'(a,b)$\\
\vskip 0.3cm
For example, $(1,1,1,0,1,1)$ can be reduced into $(1,1,2,1)$.   
\end{itemize}
\end{frame}
%%%%%%%%%%%%%%%%%%%%%%%%%%%%%%%%%%%%%%
\begin{frame}{A naive algorithm}
\begin{block}{Algorithm (Recursive method)}
 Repeat the following formula until all $\epsilon'$ belongs to ${\mathbb{Z}}^{2}$. \\
\begin{itemize}
    \item For $m \in {\mathbb{Z}}^{2r}$ with $r>1$, we have
\begin{equation*}
\TR(m)
= \tau_r(z) + \sum_{{\epsilon \in {(\{0,1\}}^{2r}}\setminus\mathbf{1})} (-1)^{\norm{\epsilon}_1}\textbf{B}(\mathbf{\mathbf{m} + \epsilon - \mathbf{1}})\TR(\epsilon') . 
\end{equation*}
where $\epsilon' \in {\{0,1\}}^{2r-2}$ and $\epsilon'(a,b)=\epsilon(a,b)$.
\vskip 0.5cm
\item For $m=(p,q) \in \mathbb{Z}^2 $, 
\begin{equation*}
 \TR(m) = \beta_{p}(x)\beta_{q}(y)z - x\beta_{p}(x)\beta_{q-1}(y) - y\beta_{p-1}(x)\beta_{q}(y) + 2\beta_{p-1}(x)\beta_{q-1}(y).
\end{equation*}
\end{itemize}
\vskip 0.5cm
\end{block}
\end{frame}
%%%%%%%%%%%%%%%%%%%%%%%%%%%%%%%%%
\subsection{Jorgensen's formular}
\begin{frame}{Jorgensen's formular}
\begin{itemize}
    \item   $\mu \in {\{-1,0,1\}}^{2r}$ is \emph{alternating} if it has even number of non-zero entries appearing alternately with sign. 
    \vskip 0.5cm
    \item For an alternating vector $\mu \in {\{-1,0,1}\}^{2r}$,
\begin{equation*}
    d(\mu) := \text{the degree of } \TR\big(\mathbf{1}-\operatorname{abs}{\mu}\big)(0,0,z)
\end{equation*}
\end{itemize}
\vskip 0.3cm
    $\mu = (0,1,0,-1,0,1,-1,0,1,-1,1,-1)$
\pause
\begin{itemize}
\setbeamertemplate{itemize item}[boldarrow]
    \item $
\TR\big(\mathbf{1}-\operatorname{abs}{\mu}\big)= \TR{(a^3b)}=x^2z-xy-z$
\vskip 0.2cm
    \item $d(\mu) = 1$. 
\vskip 0.2cm
\end{itemize}
\end{frame}
%%%%%%%%%%%%%%%%%%%%%%%%%%%%%%%%%
\begin{frame}{Jorgensen's formular}
Jorgensen presented an \emph{alternating} formula of trace polynomials.
\begin{theorem}[Jorgensen]
Let $\mathbf{m} \in \mathbb{Z}^{2r}$ for $r>0$. Then
\begin{equation*}
\label{jorgensen}
\TR(\mathbf{m})(x,y,z) = \frac{1}{2}\mathbf{B}(\mathbf{m})(x,y)\cdot \tau_r(z)+ \frac{1}{2}\sum_{\mu}\{{(-1)^{r-d(\mu)}}\cdot \mathbf{B}(\mathbf{m}+\mu)(x,y) \cdot {\tau_{d(\mu)}(z)}\}
\end{equation*}
where $\mu$ runs through all alternating $2r$-vectors.
\end{theorem}
\end{frame}
%%%%%%%%%%%%%%%%%%%%%%%%%%%%%%%%%%
%원래 책의 exercise내용과 무엇이 다른지?
%%%%%%%%%%%%%%%%%%%%%%%%%%%%%%%%%
\begin{frame}{Jorgensen's formular}
\begin{lemma}
Let $\mu$ be an alternating $2r$-vector with $\mu_1 = 0$.
Collapse all alternating subtuples of length two from $\mu$ until it is not possible. 
Let $\mu_r$ be the \textbf{\textit{reduced}} tuple after collapsing
and let $\{a_i\}_{i=0}^n$ be the sequence of the number of consecutive zeros in $\mu_r$. Then we have
\begin{align*}
     d(\mu) = \Big| \frac{1}{2}\sum_{i=0}^{n}{(-1)^i a_i} \Big|
\end{align*}
\end{lemma}
\vskip 0.3cm
$\mu = (0,1,0,-1,0,\textcolor{blue}{1,-1},0,\textcolor{blue}{1,-1,1,-1})$
\pause
\begin{itemize}
\setbeamertemplate{itemize item}[boldarrow]
    \item $\mu_r = (\textcolor{red}{0},1,\textcolor{red}{0},-1,\textcolor{red}{0,0})$
    \item $d(\mu) = \frac{1}{2}|1 -1 + 2| = 1$
\end{itemize}
\end{frame}
%%%%%%%%%%%%%%%%%%%%%%%%%%%%%%%%%%
\begin{frame}{Implement in Python3}
\tiny{
\lstinputlisting[language=Python]{Code2.py}
}
\end{frame}
%%%%%%%%%%%%%%%%%%%%%%%%%%%%%%%%%%%%%
\begin{frame}{Implement in Python3}
\tiny{
\lstinputlisting[language=Python]{Code.py}
}
\end{frame}
%%%%%%%%%%%%%%%%%%%%%%%%%%%%%%%%%%%%%%%%%%%%%%
\begin{frame}
\begin{figure}[h]
\begin{subfigure}
\includegraphics[width=6.8cm,height=8cm]{images/computation.png}
\end{subfigure}
\begin{subfigure}
\includegraphics[width=6.8cm, height=8cm]{images/table.JPG} 
\end{subfigure}
\end{figure}
\end{frame}

%%%%%%%%%%%%%%%%%%%%%%%%%%%%%%%%%%%%
\section{Trace polynomials of cyclically reduced words in $F_2$}
\subsection{Word lengths and trace polynomials}
\begin{frame}{Word lengths and trace polynomials}
$\cw := $ \emph{the set of cyclically reduced words in $F_2$}.
\vskip 0.5cm
What word information can be recovered from trace polynomials?
\begin{block}{Lemma}
\label{same_slen}
For $w \in  \mathcal{C}$, the degree of $\operatorname{\alpha}(w)$ in $z$ equals $\left\lfloor{\dfrac{\slen{w}}{2}}\right\rfloor$.
\end{block}

\end{frame}
%%%%%%%%%%%%%%%%%%%%%%%%%%%%%%%%%%%%%%%%%%%%%%%
\begin{frame}{Word lengths and trace polynomials}
\begin{theorem}
Let $w \in \cw$ with the trace polynomial $\TR(w) \in \mathbb{Z}[x,y,z]$ of degree $k$ in $z$. Also, let $f(x,y)$ be the coefficient of $z^k$ in $\TR(w)$. Then 
\begin{align*}
    \len{w} = \operatorname{deg}{f(x,y)} + 2k.
\end{align*}
\end{theorem}
\vskip 0.5cm
$\TR{(b^3a^2bab)}=xy^3\textcolor{blue}{z^2}-2xy\textcolor{blue}{z^2}-y^4z+3y^2z++x^2z-x^2y^2z-z+xy$
\begin{itemize}
\setbeamertemplate{itemize item}[boldarrow]
\item $deg(xy^3-2xy) + 2\cdot2 = 8$
\end{itemize}
\begin{block}{Corollary}
Let $w_1$ and $w_2$ in $\cw$ have the same trace polynomials. Then $\len{w_1}$=$\len{w_2}$.
\end{block}
\end{frame}
%%%%%%%%%%%%%%%%%%%%%%%%%%%%%%%%%%%%%%%%%%%%%%%
\subsection{Trace polynomial and $\sim$ equivalence on $\cw$}
\begin{frame}{Trace polynomial and $\sim$ equivalence on $\cw$}
    Let $w = u_1 u_2 u_3 \cdots u_n \in \mathcal{C}$ with $u_i \in {\{a,b,a^{-1},b^{-1}\}}$.\\ 
\vskip 0.5cm
Three operations on $\mathcal{C}$.
\begin{itemize}
    \item $\operatorname{Lsh}(w) = u_2 u_3 \cdots u_{n}u_1$,
    \item $\operatorname{Mir}(w) = u_n u_{n-1} \cdots u_1$,\text{ and}
    \item $\operatorname{Inv}(w) = w^{-1} = {u_n}^{-1} {u_{n-1}}^{-1} \cdots {u_1}^{-1}$.
\end{itemize}
\vskip 0.5cm
Define a relation $\sim$ from above.
\begin{equation*}
    w \sim w'  \Longleftrightarrow w' = F_nF_{n-1}\dots F_1(w) \text{ for finitely many } F_i \in \{\Lsh, \Mir, \Inv\}.
\end{equation*}
\end{frame}
%%%%%%%%%%%%%%%%%%%%%%%%%%%%%%%%%%%%%
\begin{frame}{Trace polynomial and $\sim$ equivalence on $\cw$}
\begin{lemma}[Wang]
Let $w,w' \in \mathcal{C}$.\\ If $w\sim w'$, then we have $\operatorname{\alpha}(w) = \operatorname{\alpha}(w')$.
\end{lemma}
\vskip 0.5cm
    $\cwn := \text{\emph{the set of cyclically reduced words of length $n$}}$. \\
    \vskip 0.3cm
\setbeamertemplate{itemize item}[boldarrow]
    \item On which $\cwn$ does the converse of Lemma hold?  \begin{itemize}
    \vskip 0.5cm
    \item Check the injectivity of $\TR : \bigslant{\cwn}{\sim} \rightarrow  \mathbb{Z}[x,y,z]$
\end{itemize}
\end{frame}

\begin{frame}{Trace polynomial and $\sim$ equivalence on $\cw$}
\begin{theorem}[main theorem]
$\TR : \bigslant{\cwn}{\sim} \rightarrow  \mathbb{Z}[x,y,z] $
is injective if and only if $n \leq 8$
\end{theorem}

\begin{block}{Corollary}
\label{main_cor}
Let $w_1, w_2 \in \cw$ with $\lvert w_1 \rvert, \lvert {w_2} \rvert \leq 8$. The followings are equivalent.
\vskip 0.3cm
\begin{enumerate}
\item $w_1 \sim w_2$
\vskip 0.3cm
\item $\TR(w_1) = \TR(w_2)$
\end{enumerate}
\end{block}
\end{frame}


%%%%%%%%%%%%%%%%%%%%%%%%%%%%%%%%%%%%%%%%%%%%%%%%%
\begin{frame}{Trace polynomial and $\sim$ equivalence on $\cw$}
\begin{block}{Proof of the main theorem - 1}
$\TR : \bigslant{\cwn}{\sim} \rightarrow  \mathbb{Z}[x,y,z] $
is injective for $n \leq 8.$
\end{block}
    \begin{enumerate}
        \vskip 0.5cm
        \item Compute all $\TR(w)$ for $w \in \cwn$.
        \vskip 0.5cm
        \item Show that $\#(\bigslant{\cwn}{\sim}) = \#\{\TR(w) \lvert w \in \cwn\}$.
        \vskip 0.3cm
    \end{enumerate}
\end{frame}
%%%%%%%%%%%%%%%%%%%%%%%%%%%%%%%%%%%%%%%%%%%%%%%%
\begin{frame}{Example with the case $n=7$}
\begin{itemize}
$N_k :=$ the number of equivalence classes in $C_7$ with the size $k$
\vskip 0.5cm
\item[1] Using algorithm, we get $\#(\TR(C_7))=106 $. Suppose that $\#(C_7)>106$.
\pause
\begin{block}{Observations}
\begin{itemize}
    \item Let $w \in \mathcal{C}$ with a prime length $p$, then $\#[w] \in \{ 2, 2p, 4p\}.$
    \item $N_2 = 2$ for all word lengths.
\end{itemize}
\end{block}
\onslide<2->{
\item[2] $2+N_{14}+N_{28} = 106$ and $4+14N_{14}+28N_{28} = \#(C_7) = 2188$.
\vskip 0.5cm
\item[3] It follows that $N_{28} < 52$. However...
}
\end{itemize}
\end{frame}
%%%%%%%%%%%%%%%%%%%%%%%%%%%%%%%%%%%%%%%%%%%%%%%%%%%%%%%%%%%%%
\begin{frame}{52 many such words ...!!}
\begin{tabular}{llll}
$a^{-4}ba^{-1}b^{-1}$ &
$a^{-4}bab^{-1}$&
$a^{-3}ba^{-1}b^{-2}$&
$a^{-3}b^{-1}a^{-1}b^{-2}$\\
$a^{-3}bab^{-2}$&
$a^{-3}b^{-1}ab^{-2}$ &
$a^{-3}ba^{-2}b^{-1}$&
$a^{-3}b^{2}a^{-1}b^{-1}$         \\
$a^{-3}b^{2}ab^{-1}$&
$a^{-3}ba^{2}b^{-1}$&
$a^{-3}b^{2}a^{-1}b$&
$a^{-3}b^{2}ab$ \\
$a^{-2}ba^{-1}b^{-3}$&
$a^{-2}b^{-1}a^{-1}b^{-3}$&
$a^{-2}bab^{-3}$&
$a^{-2}b^{-1}ab^{-3}$\\
$a^{-2}b^{2}a^{-1}b^{-2}$&
$a^{-2}b^{2}ab^{-2}$&
$a^{-2}b^{-1}a^{2}b^{2}$&
$a^{-2}b^{-1}a^{2}b^{-2}$\\
$a^{-2}ba^{-1}b^{-1}a^{-1}b^{-1}$&
$a^{-2}bab^{-1}a^{-1}b^{-1}$&
$a^{-2}b^{-1}ab^{-1}a^{-1}b^{-1}$&
$a^{-2}ba^{-1}ba^{-1}b^{-1}$\\
$a^{-2}baba^{-1}b^{-1}$	&
$a^{-2}b^{-1}aba^{-1}b^{-1}$&
$a^{-2}b^{3}a^{-1}b^{-1}$&
$a^{-2}ba^{-1}b^{-1}ab^{-1}$\\
$a^{-2}bab^{-1}ab^{-1}$	&
$a^{-2}ba^{-1}bab^{-1}$&
$a^{-2}babab^{-1}$	&
$a^{-2}b^{3}ab^{-1}$\\
$a^{-2}bab^{-1}a^{-1}b$&
$a^{-2}baba^{-1}b$	&
$a^{-2}b^{3}a^{-1}b$&
$a^{-2}b^{3}ab$	\\
$a^{-1}b^{-1}ab^{4}$&
$a^{-1}b^{-1}ab^{-4}$&
$a^{-1}b^{-2}ab^{-3}$&
$a^{-1}b^{-2}ab^{3}$\\
$a^{-1}b^{-2}a^{-1}ba^{-1}b^{-1}$&
$a^{-1}b^{-2}a^{-1}bab^{-1}$&
$a^{-1}b^{-1}ab^{-2}a^{-1}b$&
$a^{-1}b^{-1}a^{-1}b^{-2}ab^{-1}$\\
$a^{-1}b^{-1}ab^{2}a^{-1}b$&
$a^{-1}b^{-1}ab^{-1}ab^{-2}$&
$a^{-1}b^{-1}ab^{-1}ab^{2}$	&
$a^{-1}b^{-1}a^{-1}b^{-2}ab$\\
$a^{-1}b^{-1}a^{-1}b^{-1}ab^{2}$&
$a^{-1}b^{-1}a^{-1}bab^{2}$	&
$a^{-1}b^{-1}a^{-1}b^{2}a^{-1}b$	&
$a^{-1}b^{-1}aba^{-1}b^{2}$	
\end{tabular}
\begin{itemize}
    \item $\#[w]=28$ 
    \item Not $\sim$-equivalent to each other.
\end{itemize}
\end{frame}
%%%%%%%%%%%%%%%%%%%%%%%%%%%%%%%%%%%%%%%%%%%%%%%%%
\begin{frame}{Trace polynomial and $\sim$ equivalence on $\cw$}
\begin{block}{Proof of the main theorem - 2}
$\TR : \bigslant{\cwn}{\sim} \rightarrow  \mathbb{Z}[x,y,z] $
is not injective for $n \geq 9.$
\end{block}
        Consider $f\in\opertorname{End}(F_2)$ such that $f(a)=a$ and $f(b)=bab^{-1}$. 
    \begin{enumerate}
        \vskip 0.5cm
        \item For $k\geq2$, take $u_1 = ab^ka^{-1}b  \text{ and } u_2 = a^{-1}b^kab$.
        \vskip 0.5cm
        \item $f(u_1)=aba^kb^{-1}a^{-1}bab^{-1}$ and $f(u_2)= a^{-1}ba^{k}b^{-1}abab^{-1} $
        have the same trace polynomials. 
        \vskip 0.5cm
        \item But $f(u_1) \not\sim f(u_2)$.
        \vskip 0.5cm
    \end{enumerate}
\end{frame}
%%%%%%%%%%%%%%%%%%%%%%%%%%%%%%%%%%%%%%%%%%%%%%%%%%%%
\begin{frame}{Trace polynomial and $\sim$ equivalence on $\cw$}
\begin{block}{Proof of the main theorem - 2}
$\TR : \bigslant{\cwn}{\sim} \rightarrow  \mathbb{Z}[x,y,z] $
is not injective for $n \geq 9.$
\end{block}
        Consider $f\in\opertorname{End}(F_2)$ such that $f(a)=a$ and $f(b)=bab^{-1}$. 
    \begin{enumerate}
        \vskip 0.5cm
        \item For $k\geq2$, take $u_1 = ab^ka^{-1}b  \text{ and } u_2 = a^{-1}b^kab$.
        \vskip 0.5cm
        \item $f(u_1)=aba^kb^{-1}a^{-1}bab^{-1}$ and $f(u_2)= a^{-1}ba^{k}b^{-1}abab^{-1} $
        have the same trace polynomials. 
        \vskip 0.5cm
        \item But $f(u_1) \not\sim f(u_2)$.
        \vskip 0.5cm
    \end{enumerate}
\end{frame}
%%%%%%%%%%%%%%%%%%%%%%%%%%%%%%%%%%%%%%%%%%%%%
\begin{frame}
\begin{center}
\huge{Thank you}
\end{center}
\end{frame}

\begin{frame}{An optional page}

\begin{Lemma}
For $w \in \cw$, we have $\#[w] = 2c\delta$ where
\begin{align*}
c &= \operatorname{min}\{\,c_v \mid \text{ $v \in \mathcal{L}(w)$}\,\}, \text{ and}  \\
\delta &= 
\begin{cases}
1, \text{ if $w \in \mathcal{L}(Mir(w)) \cup \mathcal{L}(Mir(w^{-1}))$}\\
2, \text{ else}
\end{cases}
\end{align*}
\end{Lemma}
 
\begin{block}{Lemma}
Let $w \in \cw$ with an even syllable length. Assume $abs(w)$ has a unique syllable $abs(w)_i$. If
\begin{equation*}
\text{If } w_{i+k} \neq w_{i-k} \text{ for some $k\in\mathbb{Z}$, }
\end{equation*} then we have $\#[w]=4c$.
\end{block}
   
\end{frame}
\end{document}
